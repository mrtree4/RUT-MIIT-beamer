\documentclass[aspectratio=169]{beamer}

\usepackage{tikz}
\usepackage{caption}
\usepackage{colortbl}
\setbeamertemplate{caption}[numbered] % Включаем нумерацию для подписей
\newbool{russian}
\booltrue{russian}  % Загружает русскоязычный логотип
\usepackage{theme/theme} % Подгружаем тему

%%% Работа с русским языком и шрифтами
\usepackage[english,russian]{babel}   % загружает пакет многоязыковой вёрстки
\usepackage[no-math]{fontspec}      % подготавливает загрузку шрифтов Open Type, True Type и др.
	\setsansfont{Liberation Sans} 
	\setmonofont{Courier New}
\usepackage{mathspec}
	\setmathsfont(Digits,Latin,Greek)[Numbers={Lining,Proportional}]{Liberation Sans}
	\setmathrm[Numbers={Lining,Proportional}]{Liberation Sans}
\uselanguage{russian}
\languagepath{russian}
\graphicspath{{images/}}  	% Папка с картинками

%%% Информация об авторе и выступлении
\title[Заголовок]{Издательская система \LaTeX{}} 
\subtitle{Статьи, ссылки, системы контроля версий}
\author[Имя автора]{Александр Сергеевич Филипченко \\ \smallskip \scriptsize 797439@edu.rut-miit.ru\\}
\institute{кафедра <<Вычислительные системы, сети и информационная безопасность>>}
\date{\today}

\begin{document}	% Начало презентации

\frame[plain]{\titlepage}	% Титульный слайд

\begin{frame}
\frametitle{План лекции}
	\begin{enumerate} 
	\item Особенности подготовки научных статей в \LaTeX{}
	\item Ссылки на элементы документа
	\item Использование системы контроля версий
 	\item Домашнее задание
\end{enumerate} 
\end{frame}

\section{Особенности подготовки научных статей в \LaTeX{}}

\begin{frame}
\frametitle{Библигорафия. Программный пакет BibLaTeX}
BibLaTeX --- менеджер библиографии.
Состоит из утилиты для работы с \texttt{.bib} файлами \textbf{biber} и пакета \textbf{biblatex}.
Алгоритм работы менеджера библиографии:
\begin{enumerate} 
\item программа \textbf{xelatex} обнаруживает ссылки (команды \texttt{\textbf{\textbackslash cite}}) и подключенные источники в формате в документе и по результатам формирует запрос;
\item программа \textbf{biber} формирует в ответ на запрос LaTeX-файл с нужными библиографическими данными;
\item программа \textbf{xelatex} выполняет проход для расстановки ссылок и добавления списка литературы в документ;
\item программа \textbf{xelatex} выполняет дополнительный проход для перерасстановки номеров страниц и внутренних ссылок в документе.
\end{enumerate} 
\end{frame}

\begin{frame}
\frametitle{Пример использования BibLaTeX}
\medskip
\begin{columns}
\column{0.5\textwidth}
\begin{figure}
\centering
\begin{tikzpicture}
\node[draw=black, line width=1pt, inner sep=0pt] {\includegraphics[width=\columnwidth]{BibLatexCode}};
\end{tikzpicture}
\caption{Пример исходного кода с использованием BibLaTeX}
\end{figure}
\column{0.5\textwidth}
\begin{figure}
\centering
\begin{tikzpicture}
\node[draw=black, line width=1pt, inner sep=0pt] {\includegraphics[width=\columnwidth]{BibLatexRes}};
\end{tikzpicture}
\caption{Результат сборки}
\end{figure}
\end{columns}
\end{frame}

\begin{frame}
\frametitle{Способы управления библиографией}
Выбор стиля осуществляется через параметр \texttt{style=\texit{stylename}} при вызове \texttt{\textbf{\textbackslash usepackage}}).
Параметр \texttt{sorting=\texit{sortname}} определяет критерий сортировки источников в библиографическом списке.
% Please add the following required packages to your document preamble:
% \usepackage[table,xcdraw]{xcolor}
% Beamer presentation requires \usepackage{colortbl} instead of \usepackage[table,xcdraw]{xcolor}
\begin{table}[]
\begin{tabular}{
>{\columncolor[HTML]{FFFFFF}}l |
>{\columncolor[HTML]{FFFFFF}}l }
{\color[HTML]{1B222C} \textbf{опция}} & {\color[HTML]{1B222C} \textbf{характеристика}}                   \\ \hline
{\color[HTML]{495365} nty}            & {\color[HTML]{1B222C} сортировка по имени, названию, году}       \\
{\color[HTML]{495365} nyt}            & {\color[HTML]{1B222C} сортировка по имени, году, названию}       \\
{\color[HTML]{495365} nyvt}           & {\color[HTML]{1B222C} сортировка по имени, году, тому, названию} \\
{\color[HTML]{495365} ...}            & {\color[HTML]{1B222C} и иные комбинации этих букв}               \\
{\color[HTML]{495365} none}           & {\color[HTML]{1B222C} сортировка по порядку цитирования}        
\end{tabular}
\end{table}
\end{frame}

\begin{frame}
\frametitle{Блоки}
	\begin{theorem}[Пифагора]
		Пифагоровы штаны во все стороны равны!!!
		Если $a$ и $b$ "--- длины катетов прямоугольного треугольника, а~$c$ "--- длина гипотенузы, то $a^2+b^2=c^2$.
	\end{theorem}

	\begin{alertblock}{Блок с красным заголовком}
		Содержимое.
	\end{alertblock}

	\begin{exampleblock}{Блок с зеленым заголовком}
		Содержимое.
	\end{exampleblock}
\end{frame}


\end{document}

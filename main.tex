\documentclass[aspectratio=169]{beamer}

\usepackage{tikz}
\usepackage{caption}
\usepackage{colortbl}
\setbeamertemplate{caption}[numbered]
\newbool{russian}
\booltrue{russian}  % Загружает русскоязычный логотип
\usepackage{theme/theme} % Подгружаем тему

%%% Работа с русским языком и шрифтами
\usepackage[english,russian]{babel}   % загружает пакет многоязыковой вёрстки
\usepackage[no-math]{fontspec}      % подготавливает загрузку шрифтов Open Type, True Type и др.
	\setsansfont{Liberation Sans} 
	\setmonofont{Courier New}
\usepackage{mathspec}
	\setmathsfont(Digits,Latin,Greek)[Numbers={Lining,Proportional}]{Liberation Sans}
	\setmathrm[Numbers={Lining,Proportional}]{Liberation Sans}
\uselanguage{russian}
\languagepath{russian}
\graphicspath{{images/}}  	% Папка с картинками

%%% Информация об авторе и выступлении
\title[Заголовок]{Издательская система \LaTeX{}} 
\subtitle{Статьи, ссылки, системы контроля версий}
\author[Имя автора]{Александр Сергеевич Филипченко \\ \smallskip \scriptsize 797439@edu.rut-miit.ru\\}
\institute{кафедра <<Вычислительные системы, сети и информационная безопасность>>}
\date{\today}

\begin{document}	% Начало презентации

\frame[plain]{\titlepage}	% Титульный слайд

\begin{frame}
\frametitle{План лекции}
	\begin{enumerate} 
	\item Особенности подготовки научных статей в \LaTeX{}
	\item Ссылки на элементы документа
	\item Использование системы контроля версий
 	\item Домашнее задание
\end{enumerate} 
\end{frame}

\section{Особенности подготовки научных статей в \LaTeX{}}

\begin{frame}
\frametitle{Библигорафия. Программный пакет BibLaTeX}
BibLaTeX --- менеджер библиографии.
Состоит из утилиты для работы с \texttt{.bib} файлами \textbf{biber} и пакета \textbf{biblatex}.
Алгоритм работы менеджера библиографии:
\begin{enumerate} 
\item программа \textbf{xelatex} обнаруживает ссылки (команды \texttt{\textbf{\textbackslash cite}}) и подключенные источники в формате в документе и по результатам формирует запрос;
\item программа \textbf{biber} формирует в ответ на запрос LaTeX-файл с нужными библиографическими данными;
\item программа \textbf{xelatex} выполняет проход для расстановки ссылок и добавления списка литературы в документ;
\item программа \textbf{xelatex} выполняет дополнительный проход для перерасстановки номеров страниц и внутренних ссылок в документе.
\end{enumerate} 
\end{frame}

\begin{frame}
\frametitle{Пример использования BibLaTeX}
\medskip
\begin{columns}
\column{0.5\textwidth}
\begin{figure}
\centering
\begin{tikzpicture}
\node[draw=black, line width=1pt, inner sep=0pt] {\includegraphics[width=\columnwidth]{BibLatexCode}};
\end{tikzpicture}
\caption{Пример исходного кода с использованием BibLaTeX}
\end{figure}
\column{0.5\textwidth}
\begin{figure}
\centering
\begin{tikzpicture}
\node[draw=black, line width=1pt, inner sep=0pt] {\includegraphics[width=\columnwidth]{BibLatexRes}};
\end{tikzpicture}
\caption{Результат сборки}
\end{figure}
\end{columns}
\end{frame}

\begin{frame}
\frametitle{Способы управления библиографией}
Выбор стиля осуществляется через параметр \texttt{style=stylename} при вызове \texttt{\textbf{\textbackslash usepackage}}).
Параметр \texttt{sorting=option} определяет критерий сортировки источников в библиографическом списке.
\begin{table}[]
	\begin{tabular}{ll}
\textbf{опция} & \textbf{характеристика}                   \\ \hline
nty            & сортировка по имени, названию, году       \\
nyt            & сортировка по имени, году, названию       \\
nyvt           & сортировка по имени, году, тому, названию \\
...            & и иные комбинации этих букв               \\
none           & сортировка по порядку цитирования        
\end{tabular}
\end{table}
\end{frame}

\begin{frame}
\frametitle{Подготовка библиографического файла}
При подготовке библиографического файла \texttt{.bib} в BibLaTeX существует несколько типов записей, каждый из которых имеет свои специфические параметры.
\begin{table}[]
	\begin{tabular}{ll}
\textbf{Тип записи} & \textbf{Характеристика} \\ \hline
@book       & Книга \\
@article    & Статья в журнале \\
@conference & Материалы конференции \\
@thesis     & Диссертация или дипломная работа \\
@report     & Технический отчет \\
@manual     & Руководство или инструкция \\
@misc       & Разное (для записей, которые не подходят под другие категории)
\end{tabular}
\end{table}
\end{frame}


\begin{frame}
\frametitle{Примеры записей в библиографическом файле}
\medskip
\begin{columns}
\column{0.5\textwidth}
\begin{figure}
\centering
\begin{tikzpicture}
\node[draw=black, line width=1pt, inner sep=0pt] {\includegraphics[width=\columnwidth]{bibBook}};
\end{tikzpicture}
\caption{Параметры библиографического описания типа <<Книга>>}
\end{figure}
\column{0.5\textwidth}
\begin{figure}
\centering
\begin{tikzpicture}
\node[draw=black, line width=1pt, inner sep=0pt] {\includegraphics[width=\columnwidth]{bibArticle}};
\end{tikzpicture}
\caption{Параметры <<Статьи>>}
\end{figure}
\end{columns}
\end{frame}

\begin{frame}
\frametitle{Работа с изображениями}
\LaTeX{} не может самостоятельно управлять изображениями, поэтому необходимо использовать пакет \textbf{graphicx}.
\begin{exampleblock}{Каталог изображений}
Команда \texttt{\textbf{\textbackslash graphicspath}} сообщает \LaTeX{}, что изображения хранятся в каталоге, имя которого передано в качестве параметра.
\end{exampleblock}

\begin{exampleblock}{Включение изображения}
Команда \texttt{\textbf{\textbackslash includegraphics}} непосредственно включает изображение в документ.
В качестве параметра ей передаётся имя файла с изображением без расширения.
Имя файла с изображением не должно содержать пробелов и многоточий.
\end{exampleblock}

\begin{exampleblock}{Позиционирование}
Управлять размерами изобращений можно при помощи параметров \textbf{scale}, \textbf{width}, \textbf{height}.
Вместо конкретных численных значений ширины можно, например, задавать размер по ширине текста через \texttt{\textbf{width = \textbackslash textwidth}}.
\end{exampleblock}
\end{frame}


\begin{frame}
\frametitle{Картинки}
\begin{exampleblock}{Позиционирование}
Управлять размерами изобращений можно при помощи параметров \textbf{scale}, \textbf{width}, \textbf{height}.
Вместо конкретных численных значений ширины можно, например, задавать размер по ширине текста через \texttt{\textbf{width = \textbackslash textwidth}}.
\end{exampleblock}
\begin{exampleblock}{Подписи}
Подписи добавляющие краткое описание к изображениям.
Вызываются командой \texttt{\textbf{\textbackslash caption}}, которой в качестве параметра передаётся непсоредственно сам текст подписи.
Подписи также поддреживают автонумерацию.
\end{exampleblock}
\end{frame}

\begin{frame}
\frametitle{Блоки}
	\begin{theorem}[Пифагора]
		Пифагоровы штаны во все стороны равны!!!
		Если $a$ и $b$ "--- длины катетов прямоугольного треугольника, а~$c$ "--- длина гипотенузы, то $a^2+b^2=c^2$.
	\end{theorem}

	\begin{alertblock}{Блок с красным заголовком}
		Содержимое.
	\end{alertblock}

	\begin{exampleblock}{Блок с зеленым заголовком}
		Содержимое.
	\end{exampleblock}
\end{frame}


\end{document}
